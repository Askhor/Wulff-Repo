\documentclass{article}

\usepackage[utf8]{inputenc}
\usepackage{unicode-helper}
\usepackage{amsmath}
\usepackage{amssymb}
\usepackage{tikz}
\usepackage{amsfonts}
\usepackage{microtype}
\usepackage[english]{babel}
\usepackage{amsthm}
\usepackage{mathtools}
\usepackage{subcaption}
\usepackage{braket}
\usepackage[pdftex]{hyperref}
\usepackage{cleveref}

\newcommand{\titlevar}{Graphic}
\newcommand{\authorvar}{Julia Meßthaler}
%\newcommand{\datevar}{}
\title{\titlevar}
\author{\authorvar}
\date{\datevar}
\hypersetup{
	pdftitle=\titlevar,
	pdfauthor=\authorvar,
	pdfcreationdate=\datevar,
}
\setlength{\parindent}{0pt}

\usetikzlibrary{shapes.geometric}

\begin{document}
	\maketitle

	\newcommand{\radius}{0.05 cm}
	\newcommand{\trirad}{0.5cm}
	\newcommand{\triwidt}{0.1cm}
	\newcommand{\figwidt}{0.4\textwidth}

	\newcommand{\myfigure}[1]{
		\begin{tikzpicture}
		\fill[color=white] (0, 0) rectangle (#1 + 1,#1 + 1);

		\draw[line width=\trirad, line join=round] (#1,#1) -- (#1,1) -- (1,#1) -- cycle;
		\draw[color=white, line width=\trirad - \triwidt, line join=round] (#1,#1) -- (#1,1) -- (1,#1) -- cycle;
		\fill[color=white] (#1,#1) -- (#1,1) -- (1,#1) -- cycle;

		\foreach \x in {1,...,#1} {
			\foreach \y in {1,...,#1} {
				\fill[] (\x,\y) circle (\radius);
			}
		}
		\end{tikzpicture}
	}

	This is the stuff:

	\begin{figure}[h]
		\centering
		\begin{subfigure}{\figwidt}
			\centering
			\myfigure{2}
			\caption{$n=2$}
		\end{subfigure}
		\begin{subfigure}{\figwidt}
			\centering
			\myfigure{3}
			\caption{$n=3$}
		\end{subfigure}
		\begin{subfigure}{\figwidt}
			\centering
			\myfigure{4}
			\caption{$n=4$}
		\end{subfigure}
		\begin{subfigure}{\figwidt}
			\centering
			\myfigure{5}
			\caption{$n=5$}
		\end{subfigure}
		\caption{The stuff}
	\end{figure}
\end{document}
